\chapter{Einleitung}
\section{Zieldefinition}
\section{Vorgehen und Kurzzusammenfassung} 

\subsection{Quellen}
Quellenangaben sind im Bereich der Open Source Software schwer zu machen. Es gibt keine einheitliche Dokumentation der Software. Häufig sind die Informationen nicht an einer zentralen Stelle vereint, sondern liegen verstreut im Internet in Foren, Blogs, Mailinglisten oder auch in den Quelltexten. Die Relevanz und die Richtigkeit einer solchen Quelle ist schwer zu bewerten, da Blogs, Mailinglisten und Foren keinen Beschränkungen unterliegen. Das heißt jeder der Willens ist zu einem Thema etwas zu schreiben, kann dies auch tun.

Die oben genannte Verstreuung birgt, neben der schwierigen Bewertbarkeit der Richtigkeit und Relevanz, ein weiteres Problem. Da sehr viele Autoren zum einem Thema etwas schreiben, werden Begriffe synonym verwendet werden, sind schwammig oder sehr doppeldeutig.

Die in dieser Arbeit verwendeten Quellen enthalten Mailinglisten- und Forenbeiträge, sowie Blogs. Diese werden mit dem Erfahrungsschatz des Autors dieser Arbeit überprüft, können aber keine absolute Richtigkeit garantieren.