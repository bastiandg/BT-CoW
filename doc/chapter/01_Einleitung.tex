\chapter{Einleitung}
\section{Zieldefinition}
\section{Vorgehen und Kurzzusammenfassung} 

\subsection{Quellen}
Diese Arbeit enthält neben den herkömmlichen Quellen auch Mailinglisten- und Forenbeiträge, sowie Blogeinträge. Hierdrauf soll im Folgenden eingegangen werden.

Quellenangaben sind im Bereich der Open Source Software schwer zu machen. Es gibt keine einheitliche Dokumentation der Software. Häufig sind die Informationen nicht an einer zentralen Stelle vereint, sondern liegen verstreut im Internet in Foren, Blogs, Mailinglisten oder auch in Manpages und den Quelltexten. Die Relevanz und die Richtigkeit einer solcher Quellen ist schwer zu bewerten, da Blogs, Mailinglisten und Foren keinen Beschränkungen unterliegen. Das heißt jeder der Willens ist zu einem Thema etwas zu schreiben, kann dies auch tun.

Die oben genannte Verstreuung birgt, neben der schwierigen Bewertbarkeit der Richtigkeit und Relevanz, ein weiteres Problem. Da sehr viele Autoren zum einem Thema etwas schreiben, werden unterschiedliche Begriffe synonym verwendet oder sind mehrdeutig.

Alle Quellen werden mit der zu Grunde liegende Erfahrung des Autors dieser Arbeit ausgewählt und überprüft, können aber aus den oben genannten Gründen keine absolute Richtigkeit für sich beanspruchen. 