
\chapter{Anhang} 
%\lstset{basicstyle=\ttfamily\fontsize{9pt}{9pt}\selectfont}

\section{Skripte/Programme}\label{skripte}
Die einzelnen Aufgaben der Verwaltungslösung werden nicht in einem einzigen Skript abgearbeitet, sondern in einzelne übersichtliche Skripte aufgeteilt. In diesem Unterkapitel werden sie aufgelistet und ihre Funktion beschrieben.

\subsection{Verwaltungsserver} 
Die Skripte des Verwaltungsservers nehmen die Verwaltungs- und Einrichtungsaufgaben wahr. Ihre Aufgaben werden hier kurz erklärt.

{ \fontsize{12.1pt}{16.2pt}\selectfont
\textbf{cow.py}
- Hauptprogramm der Verwaltungslösung: Es nimmt die Benutzereingaben entgegen und ruft die anderen Skripte auf.

\textbf{hostname.sh} 
- fragt den Rechnernamen des Virtualisierungsservers ab und trägt dort den ssh-key des Verwaltungsservers ein. 

\textbf{servercert.sh}
- erstellt ein Server-Zertifikat für den Virtualisierungsserver.

\textbf{cacert.sh}
- erstellt das CA-Zertifikat wenn es nicht vorhanden ist und überträgt den öffentlichen Schlüssel des CA-Zertifikats an den jeweiligen Virtualisierungsserver.

\textbf{clientcert.sh}
- erstellt ein Client-Zertifikat, dass dem Verwaltungsserver Zugriff auf die libvirt-Installationen der Virtualisierungsserver ermöglicht.}

\subsection{Virtualisierungsserver}
Diese Skripte liegen auf den Virtualisierungsserven. Sie werden während der Einrichtung vom Verwaltungsserver installiert (siehe \ref{einrichtung}) , um lokal abrufbar zu sein. Ihre Funktionen werden im Folgenden kurz erläutert.

\textbf{clone.py}
- Klonen der virtuellen Maschinen; Das Programm nimmt die nötigen Modifikationen an der XML-Beschreibung der Vorlage vor und klont die virtuellen Festplatten.

\textbf{maketorrent.py}
- erzeugt eine torrent-Datei und startet das Verteilen mit dem BitTorrent-Client.

\textbf{packageinstall.py}
- installiert die benötigten Software-Pakete auf dem Virtualisierungsserver.

\textbf{whoami.py}
- legt eine Konfigurationsdatei mit Informationen zu dem Virtualisierungsserver an.

\textbf{xenprep.py}
- nimmt Einstellungen an dem Xen-Daemon vor, damit libvirt den Xen-Daemon abfragen kann.
\newpage
\section{Testergebnisse}
Im Folgenden sind die Testergebnisse von bonnie++ und Iozone dargestellt. Die Werte sind gemittelt. Alle Messwerte finden sich auf der CD.
\bild{iozone-xen}{Iozone Testergebnisse für Xen}{pic:iozonexenanhang}{375px}
\bild{bonnie-xen}{Bonnie++ Testergebnisse für Xen}{pic:bonniexenanhang}{375px}
\bild{bonnie-kvm}{Bonnie++ Testergebnisse für KVM}{pic:bonniekvmanhang}{375px}
\bild{iozone-kvm}{Iozone Testergebnisse für KVM}{pic:iozonekvmanhang}{375px}
\newpage

\begin{comment}
\section{Code-Listings}
\lstinputlisting[caption=cow.py,language=Python, label=cow.py]{../trunk/cow.py}
\lstinputlisting[caption=hostname.sh,language=Python, label=cow.py]{../trunk/hostname.sh}
\lstinputlisting[caption=cacert.sh,language=Python, label=cow.py]{../trunk/cacert.sh}
\lstinputlisting[caption=clientcert.sh,language=Python, label=clientcert.sh]{../trunk/clientcert.sh}
\lstinputlisting[caption=servercert.sh,language=Python, label=servercert.sh]{../trunk/servercert.sh}

\lstinputlisting[caption=client-scripts/clone.py,language=Python, label=clone.py]{../trunk/client-scripts/clone.py}
\lstinputlisting[caption=client-scripts/maketorrent.py,language=Python, label=maketorrent.py]{../trunk/client-scripts/maketorrent.py}
\lstinputlisting[caption=client-scripts/packageinstall.py,language=Python, label=packageinstall.py]{../trunk/client-scripts/packageinstall.py}
\lstinputlisting[caption=client-scripts/whoami.py,language=Python, label=whoami.py]{../trunk/client-scripts/whoami.py}
\lstinputlisting[caption=client-scripts/xenprep.py,language=Python, label=xenprep.py]{../trunk/client-scripts/xenprep.py}
\end{comment}