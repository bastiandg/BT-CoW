\chapter{Analyse Verteilung von Images}\label{verteilung}
Der Copy-on-Write-Mechanismus benötigt immer eine Vorlage - das Masterimage. Um es auf mehreren Virtualisierungsservern nutzen zu können, muss es über das Netzwerk verteilt werden oder über ein gemeinsam genutztes Storage-Backend zur Verfügung gestellt werden. Dieses Kapitel soll Wege aufzeigen, diese Verteilung oder Bereitstellung möglichst effizient vorzunehmen.

Die Verteilungslösungen werden darauf überprüft, wie störanfällig sie sind. Ein anderer Punkt für die Entscheidungsfindung ist die benötigte Dauer der Verteilung. Außerdem wird einbezogen, wie skalierbar die Lösungen sind.

\section{Multicast}
Ein Multicast ist eine Mehrpunktverbindung. Der Sender schickt die Daten gleichzeitig an mehrere Empfänger. Durch das einmalige Senden an mehrere Empfänger wird Bandbreite eingespart. Die Daten werden nur an Rechner im Netz versendet, die diese auch angefordert haben, wie in Abbildung \ref{pic:multicast} schematisch dargestellt. Die Ausnahme bilden Switches, die Multicasting nicht unterstützen, sie versenden die gesendeten Daten an alle damit verbundenen Netzwerkknoten \cite{Multicasttech}. 

Da es bei den Masterimages darauf ankommt, dass sie komplett und fehlerfrei dupliziert werden, kann der Sender maximal so schnell senden, wie es der langsamste Empfänger entgegen nehmen kann.  Dadurch ist die Verwendung von Multicast, in einer heterogenen Umgebung mit einem langsamen oder weit entfernten Empfänger, sehr ineffizient. Anwendung findet Multicast heute vor allem bei der Verteilung von Multimediadaten \cite{multicastfefe}. 

\bild{multicast}{Multicast Beispiel}{pic:multicast}{375px}

\textbf{Vorteil}
\begin{itemize}
 \item Sehr hohe Geschwindigkeit durch Parallelität
\end{itemize}

\textbf{Nachteile}
\begin{itemize}
 \item Hohe Netzwerklast während der Verteilung
%  \item schlechte Skalierbarkeit
 \item Geschwindigkeitseinbruch bei heterogener Umgebung oder schlechten Netzanbindungen
\end{itemize}

\section{Netzwerkprotokoll BitTorrent}
BitTorrent ist ein Netzwerkprotokoll zum effizienten Verteilen großer Dateien oder von Sammlungen großer Dateien. Die Empfänger der Daten sind hierbei gleichzeitig auch Sender, sie werden Peers genannt \cite{btspec}. Damit wird nicht ein einziger zentraler Sender ausgelastet, sondern die Last wird auch auf alle Empfänger verteilt (zu sehen in Abbildung \ref{pic:bittorrent}). Für die Kontaktaufnahme der Peers untereinander werden sogenannte Tracker eingesetzt. Es wird mindestens ein Tracker für die Kontaktaufnahme benötigt. Aktuellere BitTorrent-Clients können aber auch trackerlos über eine verteilte Hashtabelle (engl. ``Distributed Hash Table''; DHT) andere Peers finden \cite{dhtspec}. Durch den Einsatz von DHT kann die Einrichtung eines Trackers eingespart werden. Außerdem bringt es zusätzliche Ausfallsichertheit, da die Liste der verfügbaren Peers dezentral gespeichert wird. 

\bild{bittorrent}{Bittorrent Beispiel}{pic:bittorrent}{375px}

Die zu übertragenden Daten werden nicht komplett in einem Stück übermittelt, sondern in Blöcke aufgeteilt. Bei zwischenzeitlichen Netzausfällen müssen somit auch nicht alle Daten noch einmal übertragen werden. Der BitTorrent-Client setzt nach dem Netzwerkausfall die Datenübertragung problemlos fort und muss nur gegebenenfalls die bereits übertragenen Daten einen Blockes verwerfen.

\textbf{Vorteile}
\begin{itemize}
 \item Hohe Skalierbarkeit
 \item Netzwerklast auf teilnehmende Netzwerksegmente beschränkt
 \item Sehr effizient auch in heterogenen Umgebungen
\end{itemize}

\textbf{Nachteile}
\begin{itemize}
  \item verringerte Geschwindigkeit bei asymmetrischer Upload- und Downloadgeschwindigkeit
\end{itemize}

\section{Netzwerkprotokoll NFS}
NFS (Network File System) ist ein Protokoll für das Bereitstellen von Daten über das Netzwerk. Im Gegensatz zu den beiden vorher genannten Technologien stellt das Bereitstellen einen großen Unterschied dar. Die Daten werden nicht von einem Rechner auf den anderen kopiert, sondern über das Netzwerk wie eine lokale Festplatte zur Verfügung gestellt \cite{Nfsspec}. Der Server macht hierbei eine Freigabe, die von dem Clientrechner ``gemountet'' wird \cite{nfslinux}. Die vom Clientrechner gemountete Freigabe wird in den Verzeichnisbaum eingebunden und kann wie lokales Verzeichnis angesteuert werden.

\bild{nfs}{NFS Beispiel}{pic:nfs}{375px}

\textbf{Vorteile}
\begin{itemize}
 \item Geringer Administrationsaufwand
\end{itemize}

\textbf{Nachteile}
\begin{itemize}
 \item Schlechte Skalierbarkeit, da viele von einer NFS-Freigabe gestartete virtuelle Maschinen, zu einer permanent hohen Netzwerklast führen
 \item Keine Lastenverteilung
\end{itemize}

\section{Vergleich}
Im Folgenden werden die Verteilungsalternativen in Hinsicht auf die Kriterien Skalierbarkeit, Netzwerkausfall und Geschwindigkeit untersucht werden.

\subsection{Skalierbarkeit}
Eine gute Skalierbarkeit zeichnet sich dadurch aus, dass der Aufwand nicht signifikant ansteigt oder sich verlangsamt, wenn das Masterimage an einen weiteren Virtualisierungsserver (N+1) verteilt wird. NFS zeigt dabei eine Schwäche, die Last steigt des NFS-Servers stetig mit jedem neuen NFS-Client linear an \cite{nfsharmful}. 

Der Aufwand der Verteilung per Multicast steigt bei einem zusätzlichem Empfänger nicht an. Jedoch wird die Übertragung erheblich langsamer, wenn der zusätzliche Empfänger eine langsame Verbindung zu dem Server hat. 

Der dezentrale Aufbau des BitTorrent-Netzes macht es sehr skalierbar. Jeder zusätzliche Empfänger des Masterimages, wird auch gleichzeitig zu einem Sender. Wenn die Upload- und die Downloadgeschwindigkeit bei einem zusätzlichen Peer gleich hoch sind, wird das Netz dadurch theoretisch also nicht langsamer. Das BitTorrent-Netz profitiert sogar von zusätzlichen Peers, da sie die Störanfälligkeit des Netzes verringern \cite{btscalability}.

\subsection{Störanfälligkeit}
Hier wird verglichen, wie sich der Ausfall eines Netzwerkknotens auf die Verteilung auswirken. BitTorrent ist besonders unanfällig gegen Ausfälle im Netz. Dieses wird durch die dezentrale Struktur ermöglicht. Wenn ein einzelner Netzwerkknoten ausfällt, besteht trotzdem unter den noch verfügbaren Knoten ein Netz. 
\bild{bittorrent_ausfall}{Bittorrent Netzwerkausfall}{pic:bittorrent_ausfall}{375px}

NFS und Multicast haben im Unterschied zu BitTorrent einen großen Nachteil, da die Verteilung über einen einzigen Knoten stattfindet. Der Ausfall eines bestimmten Knotens führt also zum kompletten Abbruch der Verteilung. Man nennt diesen Punkt \textit{Single Point of Failure}. 
\bild{multicast_ausfall}{Multicast Netzwerkausfall}{pic:multicast_ausfall}{300px}

Bei NFS gibt es beim Bereitstellen der Masterimages zusätzlich die Problematik, dass der Festplattenzugriff der virtuellen Maschinen von der Verfügbarkeit des NFS-Servers abhängt. Ein Ausfall führt damit zu dem Abstürzen der virtuellen Maschinen.

\bild{nfs_ausfall}{NFS Netzwerkausfall}{pic:nfs_ausfall}{375px}

\subsection{Verteilungsdauer}
Besonders hervorzuheben ist NFS, da es nicht wie BitTorrent und Multicast die Masterimages verteilt sondern bereitstellt. Dadurch benötigt es keine Zeit die Masterimages zu verteilen und kann sie direkt zur Verfügung stellen. 

Die Dauer der Übertragung ist bei Multicast vom langsamsten beteiligten Netzwerkknoten abhängig. Ideal ist es, wenn alle Empfänger und der Sender über die gleiche Download- und Upload-Bandbreite verfügen (homogene Umgebung). So kann die gleichzeitige Übertragung an alle Empfänger optimal ausgenutzt werden.

BitTorrent zeichnet sich vor allem durch aus, dass es auch gute Ergebnisse erzielt, wenn die Peers über unterschiedliche Download- und Upload-Geschwindigkeiten verfügen. In einer homogenen Umgebung benötigt es mehr Zeit für die Verteilung als Multicast. 

\section{Fazit}

Alle aufgezeigten Lösungen für das Verteilen von Masterimages haben ihre Vor- und Nachteile. Jedoch zeigt sich, dass BitTorrent wesentliche Vorteile gegenüber den anderen beiden Lösungen hat. Eine geringe Störanfälligkeit ist im produktiven Einsatz sehr wichtig. Auf diesem Gebiet liegt BitTorrent weit vor NFS und Multicast. Multicast ist ein sehr effizientes Protokoll und es wird produktiv eingesetzt, zum Beispiel bei der Übertragung von Multimediadaten. Allerdings ist es ungeeignet wenn der Empfänger alle Daten erhalten muss. Auch die Erweiterbarkeit um zusätzliche Virtualisierungsserver unterstützt die Schlussfolgerung, dass BitTorrent für den hier diskutierten Einsatz die effizienteste Lösung ist.