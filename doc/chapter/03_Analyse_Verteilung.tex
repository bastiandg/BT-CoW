\chapter{Analyse Verteilung von Images}
Der Copy-on-Write-Mechanismus benötigt immer eine Vorlage - das Masterimage. Um es auf mehreren Virtualisierungsservern nutzen zu können, muss es über das Netzwerk verteilt werden. Dieses Kapitel soll Wege aufzeigen diese Verteilung möglichst effizient vorzunehmen.

Die Verteilungslösungen werden darauf überprüft wie sie kurzfristige und langfristige Netzwerkausfälle handhaben. Ein anderer Punkt für die Entscheidungsfindung ist die Geschwindigkeit der Übertragung. Außerdem wird einbezogen wie skalierbar die Lösungen sind.

\section{Multicast}
Multicast ist eine Mehrpunktverbindung. Der Sender schickt die Daten gleichzeitig an mehrere Empfänger. Durch das einmalige Senden an mehrere Empfänger wird Bandbreite eingespart. Die Daten werden nur an Rechner im Netz versendet diese auch angefordert haben, wie in Abbildung \ref{pic:multicast} schematisch dargestellt. Die Ausnahme bilden Switches die Multicasting nicht unterstützen, sie versenden die gesendeten Daten an alle damit verbundenen Netzwerkknoten. 

Da es bei den Masterimages darauf ankommt, dass sie komplett und fehlerfrei dupliziert werden, kann der Sender maximal so schnell senden, wie es der langsamste Empfänger entgegen nehmen kann.  Dadurch ist die Verwendung von Multicast, in einer heterogenen Umgebung mit einem langsamen oder weit entfernten Empfänger, sehr ineffizient. Anwendung findet Multicast heute vor allem bei der Verteilung von Multimediadaten. 

\bild{multicast}{Multicast Beispiel}{pic:multicast}{400px}

\textbf{Vorteile}
\begin{itemize}
 \item sehr hohe Geschwindigkeit durch Parallelität
\end{itemize}

\textbf{Nachteile}
\begin{itemize}
 \item hohe Netzwerklast
%  \item schlechte Skalierbarkeit
 \item Geschwindigkeitseinbruch bei heterogener Umgebung
\end{itemize}

\section{BitTorrent}
BitTorrent ist ein Netzwerkprotokoll zum effizienten Verteilen großer Datenmengen. Die Empfänger der Daten sind hierbei gleichzeitig auch Sender, sie werden Peers genannt. Damit wird nicht ein einziger zentraler Sender ausgelastet, sondern die Last wird auch auf alle Empfänger verteilt (zu sehen in Abbildung \ref{pic:bittorrent}). Für die Kontakaufnahme der Peers untereinander wird ein sogenannter Tracker benötigt. Aktuellere BitTorrent-Clients können aber auch trackerlos über eine verteilte Hashtabelle (engl. ``Distributed Hash Table''; DHT) andere Peers finden. 

Die zu übertragenden Daten werden nicht komplett in einem Stück übermittelt, sondern in Blöcke aufgeteilt. Bei zwischenzeitlichen Netzausfällen müssen somit auch nicht alle Daten noch einmal übertragen werden. Der BitTorrent-Client setzt nach dem Netzwerkausfall die Datenübertragung problemlos fort und nur gegebenenfalls die bereits übertragenen Daten einen Blockes verwerfen.

\bild{bittorrent}{Bittorrent Beispiel}{pic:bittorrent}{400px}

\textbf{Vorteile}
\begin{itemize}
 \item hohe Skalierbarkeit
 \item niedrige Netzwerklast
 \item sehr effizient auch in heterogenen Umgebungen
\end{itemize}

\textbf{Nachteile}
\begin{itemize}
 \item geringere Geschwindigkeit
\end{itemize}

\section{NFS}
NFS (Network File System) ist ein Protokoll für das Bereitstellen von Daten über das Netzwerk. Das ist ein großer Unterschied zu den beiden vorher genannten Technologien. Die Daten werden nicht von einem Rechner auf den anderen kopiert, sondern über das Netzwerk wie eine lokale Festplatte zur Verfügung gestellt. Der Server macht hierbei eine Freigabe die von dem Clientrechner ``gemountet'' wird. Die vom Clientrechner gemountete Freigabe wird in der Verzeichnisbaum eingebunden und kann wie lokales Verzeichnis angesteuert werden.

\bild{nfs}{NFS Beispiel}{pic:nfs}{400px}

\textbf{Vorteile}
\begin{itemize}
 \item geringer Einrichtungsaufwand
\end{itemize}

\textbf{Nachteile}
\begin{itemize}
 \item schlechte Skalierbarkeit
 \item viele von einer NFS-Freigabe gestarte virtuelle Maschinen, können zu einer permanent hohen Netzwerklast führen
 \item schlechte Lastenverteilung
\end{itemize}

\section{Fazit}

Alle aufgezeigten Lösungen für das Verteilen von Masterimages haben ihre Vor- und Nachteile. Wie die Tests gezeigt haben sind alle 3 für den produktiven Einsatz geeignet. Jedoch zeigt sich das BitTorrent wesentliche Vorteile gegenüber den anderen beiden Lösungen hat. Es lässt sich im Gegensatz zu Multicast auch in heterogenen Umgebungen einsetzen und ist sehr viel skalierbarer als NFS.