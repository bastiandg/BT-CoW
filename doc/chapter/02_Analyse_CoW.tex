\chapter{Analyse Copy-on-Write}
\section{Sparse-Dateien}
Eine Sparse-Datei ist eine Datei, die nicht vom Anfang bis zum Ende beschrieben ist. Sie enthält also Lücken. Um Speicherplatz zu sparen, werden diese Lücken bei Sparse-Dateien nicht auf den Datenträger geschrieben. Eine Sparse-Datei ist kein eigenes Imageformat sondern eine Optimierungsstrategie. \cite{sparse}

\bild{Sparse}{Sparse-Datei}{pic:sparse}{400px}

\section{qcow2}
Das Imageformat qcow2 ist im Rahmen des qemu Projekts entwickelt wurde. Es ist der Nachfolger des ebenfalls aus dem qemu Projekt stammenden Formats qcow. \cite{qcowmarkmc} \cite{qemuwiki}

\subsection{Vorteile}
\begin{itemize}
 \item einfache Einrichtung
\end{itemize}

\subsection{Nachteile}
\begin{itemize}
 \item Kompatibilitätsprobleme mit Xen und anderen offenen Virtualisierungstechniken (z.B. VirtualBox)
\end{itemize}

\section{vhd}
Das Format vhd ist von Conectix und Microsoft entwickelt worden. Die Spezifikation des Imageformats wurde von Microsoft im Zuge des ``Microsoft Open Specification Promise'' freigegeben. Seit der Freigabe der Spezifikation bieten einige Open Source Virtualisierungstechnologien wie qemu, Xen oder VirtualBox die Möglichkeit dieses Format zu verwenden. \cite{msosp} \cite{vhdspec} %\gls{test}

\subsection{Vorteile}
\begin{itemize}
 \item einfache Einrichtung
\end{itemize}

\subsection{Nachteile}
\begin{itemize}
 \item Weiterentwicklung ist fragwürdig
 \item Verwendung von Copy on Write mit KVM nicht möglich
\end{itemize}

\section{dm-snapshots}
Die dm-snapshots sind eine Funktion des Device Mappers. Device Mapper ist ein Treiber im Linux-Kernel. Er erstellt virtuelle Gerätedateien, die mit bestimmten Features wie zum Beispiel Verschlüsselung ausgestattet sind. Bei dm-snapshots wird eine solche virtuelle Gerätedatei erstellt. Sie wird aus zwei anderen Gerätedateien zusammengesetzt. Die erste Gerätedatei ist der Ausgangspunkt, wenn an daran Änderungen vorgenommen werden, werden sie als Differenz in der zweiten Gerätedatei gespeichert. \cite{dmmbroz} \cite{dmkerneldoc}

\subsection{Vorteile}
\begin{itemize}
 \item hohes Entwicklungsstadium
 \item sichere Weiterentwicklung
\end{itemize}

\subsection{Nachteile}
\begin{itemize}
 \item Aufwendige Einrichtung
 \item unabhängig von Virtualisierungstechnik
\end{itemize}

\section{LVM-Snapshots}
LVM-Snapshots sind ein Teil des Logical Volume Managers. LVM ist eine Software-Schicht die über den eigentlichen Hardware-Festplatten einzuordnen ist. Es basiert auf Device Mapper. LVM ermöglicht das Anlegen von virtuelle Partitionen (logical volumes). Diese können sich über mehrere Festplatten-Partitionen erstrecken und Funktionen wie Copy-on-Write bereitstellen. \cite{lvmhowto} \cite{lvmselflinux} \cite{lvmsource}

\subsection{Vorteile}
\begin{itemize}
 \item hohes Entwicklungsstadium
 \item sichere Weiterentwicklung
 \item unabhängig von Virtualisierungstechnik
\end{itemize}

\subsection{Nachteile}
\begin{itemize}
 \item Aufwendige Einrichtung
% \item Migration schwierig
 \item Live-Migration nicht möglich
 \item Nutzung von Sparse-Dateien schwer umsetzbar
\end{itemize}

\section{Benchmarks}
Ein wichtiger Punkt für die Entscheidung welche Copy-on-Write Implementierung optimal ist, ist die Lese- und Schreibgeschwindigkeit. Hierbei gibt es zwei Zugriffsarten, einmal den sequentiellen Zugriff und den wahlfreien oder auch zufälligen Zugriff. Die Testergebnisse werden in diesem Kapitel zusammenfassend aufgeführt. Die kompletten Testergebnisse befinden sich im Anhang.

\subsection{IOzone}
IOzone ist ein Tool mit dem in einer Reihe von unterschiedlichen Tests die Lese- und Schreib-Geschwindigkeit überprüft werden kann. Es wird hier zur Überprüfung der sequentiellen Lese- und Schreibgeschwindigkeit verwendet.
\subsection{Bonnie++}
Bonnie++ dient wie IOzone als Tool zum Testen von Festplatten. Es wird hier zur Überprüfung der sequentiellen Lese- und Schreibgeschwindigkeit sowie zum Testen des wahlfreien Zugriffs verwendet.